\documentclass{beamer}

\usepackage{amsmath}
\usepackage{tikz-cd}
\usepackage[T1]{fontenc}
\usepackage[utf8]{inputenc}
\usepackage{listings}

\usepackage{color}
\definecolor{keywordcolor}{rgb}{0.7, 0.1, 0.1}   % red
\definecolor{tacticcolor}{rgb}{0.0, 0.1, 0.6}    % blue
\definecolor{commentcolor}{rgb}{0.4, 0.4, 0.4}   % grey
\definecolor{symbolcolor}{rgb}{0.0, 0.1, 0.6}    % blue
\definecolor{sortcolor}{rgb}{0.1, 0.5, 0.1}      % green
\definecolor{attributecolor}{rgb}{0.7, 0.1, 0.1} % red

\def\lstlanguagefiles{lstlean.tex}
% set default language
\lstset{language=lean}

\usetheme{Ilmenau}

\title{Lawvere's Fixed Point Theorem}
\author{Johannes Folttmann}
\institute{P4A1}
\date{January 31, 2025}

\AtBeginSection[]
{
  \begin{frame}
    \frametitle{Table of Contents}
    \tableofcontents[currentsection]
  \end{frame}
}

\begin{document}

\frame{\titlepage}

\section{Cartesian Closed Categories}

\begin{frame}
  \frametitle{Definition of Cartesian Closed Categories (CCCs)}
  \begin{block}{Definition}
    A Category $\mathcal{C}$ is \textbf{Cartesian Closed} if it has (chosen) finite products and for all $A \in \text{ob}(\mathcal{C})$ the functor $- \times A$ has a right adjoint $(-)^{A}$.
  \end{block}

  For $X, Y, Z \in \text{ob}(\mathcal{C})$, there is  a natural bijection
  \[\text{curry} : \text{hom}(X \times Y, Z) \rightarrow \text{hom}(X, Z ^ Y)\]

  We also get the following evaluation map:
  \[\text{ev}_{X,Y} = \text{curry}^{-1} (\text{id}_{X ^ Y}) : X ^ Y \times Y \rightarrow X\]
\end{frame}

\begin{frame}[fragile]
  \frametitle{Implementation in Mathlib}
  In Mathlib CCCs are definded as monoidal closed categories, where the monoidal structure $(C, \top, \times)$ is given the Cartesian product:
  \begin{lstlisting}
  abbrev CartesianClosed (C : Type u) [Category.{v} C] [ChosenFiniteProducts C] :=
    MonoidalClosed C
  \end{lstlisting}
  Here \texttt{MonoidalClosed} is defined as follows:
  \begin{lstlisting}[escapeinside={(*}{*)}]
  class Closed {C : Type u} [Category.{v} C] [MonoidalCategory.{v} C] (X : C) where
    rightAdj : C (*$\Rightarrow$*) C
    adj : tensorLeft X (*$\dashv$*) rightAdj
  class MonoidalClosed (C : Type u) [Category.{v} C] [MonoidalCategory.{v} C] where
    closed (X : C) : Closed X := by infer_instance
  \end{lstlisting}

\end{frame}

\section{Lawvere's Fixed Point Theorem(s)}

\begin{frame}[fragile]
  \frametitle{(Weak) Point Surjectivity}
  \begin{block}{Definition}
    A morphism $\Phi : X \rightarrow Y$ is \textbf{point surjective} (ps), if for every ``point'' $x : \top \rightarrow X$, there exists a ``point'' $y : \top \rightarrow Y$, such that $\Phi (x) = y$.
  \end{block}
  \begin{block}{Definition}
    A morphism $\Phi : X \rightarrow Z ^ Y$ is \textbf{weakly point surjective} (wps), if for every $g : Y \rightarrow Z$, there exists a ``point'' $x : \top \rightarrow X$, such that for all $y : \top \rightarrow Y$ we have $g(y) = \Phi(x)(y)$.
    \end{block}
    Here $\Phi (x)$ denotes $\Phi \circ x$ and $\Phi (x)(y)$ denotes the morphism
    \[\begin{tikzcd}
	    \top & {Z^Y \times Y} & Z
	    \arrow["{\langle \Phi \circ x, y \rangle}", from=1-1, to=1-2]
	    \arrow["{\text{ev}_{Z,Y}}", from=1-2, to=1-3]
    \end{tikzcd}\]
\end{frame}

\begin{frame}
  \frametitle{Lawvere's fixed point theorem (LFPT)}
  \begin{alertblock}{Theorem}
    Let $\mathcal{C}$ be a CCC. If there is a weakly point surjective morphism $\Phi : A \rightarrow B ^ A$ in $\mathcal{C}$, then every morphism $f : B \rightarrow B$ has a fixed point (a $s : \top \rightarrow B$ such that $f \circ s = s$).
  \end{alertblock}
  \begin{alertblock}{Theorem (alternative version)}
    Let $\mathcal{C}$ be a category with finite products. If there is a ``weakly point surjective'' morphism $\Phi : A \times A \rightarrow B$ in $\mathcal{C}$, then every morphism $f : B \rightarrow B$ has a fixed point (a $s : \top \rightarrow B$ such that $f \circ s = s$).
  \end{alertblock}
\end{frame}

\begin{frame}[fragile]
  \frametitle{Proof of LFPT}
  \begin{proof}
    Let $g : A \rightarrow B$ be the composite morphism
    \[\begin{tikzcd}
	    A & {A \times A} & {B^A\times A} & {B} & {B}
	    \arrow["\Delta", from=1-1, to=1-2]
	    \arrow["{\Phi \times \text{id}_A}", from=1-2, to=1-3]
	    \arrow["{\text{ev}_{B,A}}", from=1-3, to=1-4]
      \arrow["f", from=1-4, to=1-5]
    \end{tikzcd}\]
    Because $\Phi$ is wps, there exists a $p : \top \rightarrow A$ such that $g = \Phi(p)$. Now $\Phi(p)(p) = g(p)$ is the morphism
    \[\begin{tikzcd}
	    \top & {B^A \times A} & {B} & {B}
	    \arrow["{\langle \Phi \circ p, p \rangle}", from=1-1, to=1-2]
	    \arrow["{\text{ev}_{B,A}}", from=1-2, to=1-3]
      \arrow["f", from=1-3, to=1-4]
    \end{tikzcd}\]
  Thus $\Phi(p)(p) = f \circ \Phi(p)(p)$.
  \end{proof}
\end{frame}

\section{Application to self referential theorems}

\begin{frame}
  \frametitle{Self-referential Theorems}
  We can apply LFPT to several ``self-referential'' theorems in logic in the following way:
  \begin{itemize}
    \item For any $L$-theory $T$ we construct a category with finite products and designated elements $\Omega$ and $A$.
    \item Morphisms $t : \top \rightarrow A^n$ correspond to classes of constant terms.
    \item Morphisms $\varphi : A^n \rightarrow \Omega$ correspond to classes of formulas with $n$ free variables.
    \item Composition corresponds to substitution.
    \item There are morphisms $\text{true, false} : \top \rightarrow \Omega$
    \item A Gödel numbering can be regarded as a point surjection $g : A \rightarrow \Omega ^ A$ ($\Omega ^ A$ does not exist, but we can use currying)
  \end{itemize}
\end{frame}

\end{document}